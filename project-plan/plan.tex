\documentclass{wx672article} % $HOME/texmf/tex/latex/wx672article.cls

% wx672{common,fonts,bib} loaded in wx672a.sty
\usepackage{wx672cjk}

\title{项目计划书}
\author{蒲启元 \\
\emph{pqy7172@gmail.com}}

\begin{document}

\maketitle{}

\section{执行总结}

\subsection{项目背景}
国内尚无十分成熟的自主操作系统。操作系统开发的核心技术都掌握在外国人手中。《中国制造2025》
特地提出了要重视操作系统的开发。

操作系统理论知识已经具备,开发平台也已搭建完成。

\subsection{项目规划}

前期是平台的搭建,Boot Loader的编写,中期是内存管理,进程管理,图形管理等,后期是几个应用
开发。

\section{项目简介}

\subsection{概述}

操作系统管理着计算机的硬件和软件资源,它是向上层应用软件提供服务(接口)的核心系
统软件,这些服务包括进程管理,内存管理,文件系统,网络通信,安全机制等。操作系统的设计与
实现则是软件工业的基础。为此,在国务院提出的《中国制造2025》中专门强调了操作系统的开
发。但长期以来,操作系统核心开发技术都掌握在外国人手中,技术受制,对于我
们的软件工业来说很不利。本项目从零开始设计开发一个简单的操作系统,包括boot loader,中断,
内存管理,图形接口,多任务等功能模块,以及能运行在这个系统之上的几个小应用程序。尽管这
个系统很简单,但它是自主开发操作系统的一次尝试。

\subsection{发展规划}

完全从零开发通用操作系统其可行性并不大,我们的发展望是未来至少能将这个操作系统用于教学。使
得抽象的操作系统概念能够得以具体化,从而理解更加深化,在这个基础上,再吸纳同学加入到这里面
继续开发。

\section{团队介绍}

\subsection{蒲启元}
西南林业大学计算机科学与技术专业学生。热爱操作系统开发。校级、省级三好学生,优秀毕业生,优
秀毕业论文(设计),泰国交换生。Github主页:https://github.com/Puqiyuan。

\subsection{朱卓}

西南林业大学计算机科学与技术专业学生。朱卓,男,21岁,西南林业大学计算机网络大四学生,擅长
C编程,Linux操作系统,英文能力较强。曾赴泰国西那瓦国际 大学交流学习半年。在校期间多次获得
奖学金。

\subsection{李政荣}
西南林业大学计算机科学与技术专业学生。热爱操作系统开发。

\end{document}

%%% Local Variables:
%%% mode: latex
%%% TeX-master: t
%%% End:
