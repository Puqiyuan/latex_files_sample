\documentclass[UKenglish]{beamer}


\usetheme{UiB}


\usepackage[utf8]{inputenx} % For æ, ø, å
\usepackage{csquotes}       % Quotation marks
\usepackage{microtype}      % Improved typography
\usepackage{amssymb}        % Mathematical symbols
\usepackage{mathtools}      % Mathematical symbols
\usepackage[absolute, overlay]{textpos} % Arbitrary placement
\setlength{\TPHorizModule}{\paperwidth} % Textpos units
\setlength{\TPVertModule}{\paperheight} % Textpos units
\usepackage{tikz}
\usepackage{minted}
\usetikzlibrary{overlay-beamer-styles}  % Overlay effects for TikZ


\author{Cauchy Pu}
\title{Assembly}
\subtitle{Where we stand and where we go}


\begin{document}


% Use
%
%     \begin{frame}[allowframebreaks]{Title}
%
% if the TOC does not fit one frame.

\begin{frame}
\frametitle{Outline}
\tableofcontents
\end{frame}

\section{Overview}
\begin{frame}{Overview}
  Computers execute machine code, sequences of bytes encoding the low-level operations
  that manipulate data, manage memory, read and write data on storage devices, and
  communicate over networks.

  \begin{alertblock}{Bits and Where}
   Information = Bits + Context
 \end{alertblock}

 \begin{center}
   \structure{So we study what bits and in which we view them}\\ \\
 \end{center}

 \begin{center}
   \vspace{1cm}
   \Huge Assembly!
 \end{center}
\end{frame}


\section{Why assembly}
\begin{frame}{Why assembly}
  But, compilers do most of the work in generating assembly code, so why sould we spend
  our time?
  \begin{enumerate}
  \item Optimization. Sometimes we have to try the assembly code corresponding to various
    forms of upper language.
  \item Some bugs more obvious in assembly. For exmaple, concurrent programs.
  \item Security. Overwrite information is a common attack method.
  \item Some code must be assembly, context switch.
  \item When you encounter core dump, what the corresponding high level code?
  \item And you told me, we are low-level engineer, right? So, here we go, assembly!
  \item ......
  \end{enumerate}
   \begin{center}
   \Huge Think like a computer
 \end{center}
\end{frame}

\begin{frame}{Why assembly}
  \begin{example}
    \includegraphics[width = \textwidth, height=2cm]{canary.png}
    \includegraphics[width = \textwidth, height=4.5cm]{canary_assembly.png}
  \end{example}
\end{frame}

\begin{frame}{Why assembly}
  \begin{example}
    \includegraphics[width = \textwidth, height=6.5cm]{context_switch.png}
  \end{example}
\end{frame}

%high language how to map to assembly, instructions
\section{A simple world}
\begin{frame}{A simple world}
  In the world of assembly...
  \includegraphics[width = \textwidth, height=4.5cm]{simple.png}
  \begin{center}
   \Huge Simple, right?
  \end{center}
\end{frame}

\begin{frame}{A simple world}
  The class of instructions:
  \begin{enumerate}
  \item assessing
    \begin{itemize}
    \item mov
    \item ldl
    \item stl
    \end{itemize}
  \item arithmetic and logical operations
    \begin{itemize}
  \item add
  \item sub
    \end{itemize}
  \item control \& procedures
    \begin{itemize}
  \item jump
  \item call
  \item ret
    \end{itemize}
  \end{enumerate}
\end{frame}

\begin{frame}{A simple world}
  How stack operations...
  \begin{center}
    \includegraphics[width = 0.8\textwidth, height=7.5cm]{stack.png}
  \end{center}
\end{frame}

\begin{frame}{A simple world}
  I just, want struct, array and many functions...
    \begin{center}
    \includegraphics[width = 0.8\textwidth, height=4.5cm]{addressing.png}
  \end{center}
  Complete example:
  \begin{center}
    rax = rdx + 4 * rcx - 12\\
   This is the so-called \color{red}{addressing mode}
  \end{center}
\end{frame}

  \begin{frame}{A simple world}
    I just, want struct, array and many functions... So, our Starring!
\begin{columns}[onlytextwidth]
    \begin{column}{0.45\textwidth}
          \begin{minipage}[c][0.56\textheight][c]{\linewidth} 
          \includegraphics[width = 0.8\textwidth, height=4.5cm]{array.png}
          \end{minipage}
    \end{column}
    \begin{column}{0.5\textwidth}
      \begin{minipage}[c][0.55\textheight][c]{1\linewidth} 
        \begin{block}{How calculation}
          T A[R][C]\\
          \&A[i][j] = x_A + L(C * i + j)\\
          L: the\enspace size\enspace of\enspace element\enspace of\enspace D
         \end{block}
         \end{minipage}
       \end{column}
     \end{columns}
     \includegraphics[width = 0.8\textwidth, height=1.5cm]{array_assembly.png}
     \begin{center}
         \\\Large  Really simple, right? ;-)
     \end{center}
   \end{frame}

   \begin{frame}{A simple world}
     Eventually, it's the mysterious of function call!
     \begin{center}
       \includegraphics[width = 0.8\textwidth, height=3.5cm]{function_tr.png}
       \includegraphics[width = 0.8\textwidth, height=3.5cm]{function_call.png}
     \end{center}
   \end{frame}
   
   

\begin{frame}{A simple world}
  Now some real feed...
  \begin{example}
    .section .data\\
    data\_items:\\
    .long 3,67,34,222,45,75,54,34,44,33,22,11,66,0\\
    .section .text\\
    .global \_start\\
    \_start:\\
    movl \$0, \$edi\\
    movl data\_items(,\%edi, 4), \%eax\\
    movl \%eax, \%ebx\\

    start\_loop:\\
    cmpl \$0, \$eax\\
  \end{example}
\end{frame}

\begin{frame}{A simple world}
  \begin{example}
    je loop\_exit\\
    incl \%edi\\
    movl data\_items(,\$edi, 4), \%eax\\
    cmpl \%ebx, \%eax\\
    jle start\_loop\\
    movl \%eax, \%ebx\\
    jmp start\_loop\\
    loop\_exit:\\
    movl \$1, \%eax\\
    int \$0x80\\
  \end{example}
  \begin{enumerate}
    \pause
  \item Why cmpl \$0, \$eax? \pause \color{red}{the last item of data\_items}
    \pause
  \item \color{black}{How to check result?} \pause \color{red}{echo \$?}
    \pause
  \item \color{black}{How to pass the result to next function?} \pause \color{red}{Euh...next section}
  \end{enumerate}
\end{frame}

\begin{frame}{A simple world}
  Some \color{red}{caveat}
  \begin{enumerate}
  \item SP is a freak!
  \item You only have a limited number of registers.
  \item Frame, frame, it's frame...
  \item Alignment encounter with SP.
  \item ......
  \end{enumerate}
  \begin{center}
    \includegraphics[width = 0.2\textwidth, height=2.5cm]{grimace.jpg}
  \end{center}
  \begin{center}
    \Large \emph{Simple but frustrated...}
  \end{center}
\end{frame}

%abi
\section{Agreements(ABI)}
\begin{frame}{Agreements(ABI)}
  Why Application Binary Interface(ABI)?\\

 \begin{alertblock}{Definition}
In computer software, an application binary interface (ABI) is an interface between two binary program modules
\end{alertblock}
ABIs cover details such as:
\begin{enumerate}
\item a processor instruction set (with details like register file structure, stack
  organization, memory access types, ...)
\item the sizes, layouts, and alignments of basic data types that the processor can directly access
\item the calling convention
\item syscall and more\\
  \begin{center}
    \Large Something like TCP/IP?
  \end{center}
  
\end{enumerate}
\end{frame}

\begin{frame}{Agreements(ABI)}
Our interest, the calling convention
\begin{alertblock}{Definition}
In computer science, a calling convention is an implementation-level (low-level) scheme
for how subroutines receive parameters from their caller and how they return a result.
\end{alertblock}
\begin{center}
  So, for X86-64, what the convention?
  \begin{columns}[onlytextwidth]
    \begin{column}{0.45\textwidth}
          \begin{minipage}[c][0.56\textheight][c]{\linewidth} 
          \includegraphics[width = 1\textwidth, height=3.5cm]{param.png}
          \end{minipage}
    \end{column}
    \begin{column}{0.5\textwidth}
      \begin{minipage}[c][0.55\textheight][c]{1\linewidth}
        \begin{block}{Notes}
        \item 1, rax is the result
        \item 2, arguments on stack if more than six
        \end{block}
      \end{minipage}
      
       \end{column}
     \end{columns}
\end{center}

\end{frame}

\begin{frame}{Agreements(ABI)}
  Some examples...
  
\begin{center}
  \begin{columns}[onlytextwidth]
    \begin{column}{0.45\textwidth}
          \begin{minipage}[c][0.56\textheight][c]{\linewidth} 
          \includegraphics[width = 1\textwidth, height=3.5cm]{proc_c.png}
          \end{minipage}
        \end{column}

        \begin{column}{0.45\textwidth}
          \begin{minipage}[c][0.56\textheight][c]{\linewidth} 
          \includegraphics[width = 1\textwidth, height=3.5cm]{caller_c.png}
          \end{minipage}
    \end{column}
  \end{columns}

  \Large Imagine what the assembly code?

  \begin{center}
    \includegraphics[width = 0.1\textwidth, height=1cm]{think.png}
  \end{center}
\end{center}

\end{frame}

\begin{frame}{Agreements(ABI)}
  \vspace{1.2cm}
\begin{center}
  \begin{columns}[onlytextwidth]
    \begin{column}{0.45\textwidth}
          \begin{minipage}[c][0.56\textheight][c]{\linewidth} 
          \includegraphics[width = 1\textwidth, height=8cm]{proc_assembly.png}
          \end{minipage}
        \end{column}

        \begin{column}{0.45\textwidth}
          \begin{minipage}[c][0.56\textheight][c]{\linewidth} 
          \includegraphics[width = 1\textwidth, height=8cm]{caller_assembly.png}
          \end{minipage}
    \end{column}
  \end{columns}
\end{center}

\end{frame}

\section{How many architectures}
\begin{frame}{How many architectures}
  So, each architecture corresponds to an ABI. How many architectures theres is?

  \begin{center}
    \color{red}{Well, Of course, I don't know ;-)}
  \end{center}

  But for common architecture:
  \begin{enumerate}
  \item arm32
    \begin{enumerate}
    \item arguments: r0-r3
    \item results: the same as arguments
    \end{enumerate}
  \item X86
    \begin{enumerate}
    \item arguments: it depends. On the stack?
    \item results: eax
    \end{enumerate}
  \item SW64(Yes, your old friend!)
    \begin{enumerate}
    \item arguments: r16-r21
    \item results: r0 \& r1
    \end{enumerate}
  \end{enumerate}
\end{frame}

\section{Inline Assembly}
\begin{frame}{Inline Assembly}
  When you travel happily in the kernel source...
  \begin{alertblock}{Damn!}
    \_\_asm\_\_ ("swp \%0, \%0, [\%1]" : : "r"(val), "r"(addr));
  \end{alertblock}
  This is inline assembly, a syntax that allows you to write assembly in C(or other high
  level language?)
  \begin{center}
    \begin{example}
       int a=10, b;\\
        asm ("movl \%1, \%\%eax; \\
              movl \%\%eax, \%0;"\\
             :"=r"(b)        /* output */\\
             :"r"(a)         /* input */\\
             :"\%eax"         /* clobbered register */\\
             );  
    \end{example}
  \end{center}
  
\end{frame}

\section{Reverse Engineering}
\begin{frame}{Reverse Engineering}
  Some friends you need...
  \begin{enumerate}
  \item man objdump
  \item man readelf
  \item man gdb
  \item man gcc
  \item ELF(a really big topic)
  \item last but not least, Google
        % TODO to explain switch case
  \end{enumerate}
\end{frame}

\section{Reverse Engineering}
\begin{frame}{Reverse Engineering}
  Try the real game!
  \vspace{1cm}
\begin{center}
  \begin{columns}[onlytextwidth]
    \begin{column}{0.45\textwidth}
          \begin{minipage}[c][0.56\textheight][c]{\linewidth} 
          \includegraphics[width = 1\textwidth, height=6.5cm]{switch_assembly.png}
          \end{minipage}
        \end{column}

        \begin{column}{0.45\textwidth}
          \begin{minipage}[c][0.56\textheight][c]{\linewidth} 
          \includegraphics[width = 1\textwidth, height=6.5cm]{switch_c.png}
          \end{minipage}
    \end{column}
  \end{columns}
\end{center}
\end{frame}

\section{Questions}
\begin{frame}{Questions}
  \begin{center}
    \vspace{1.5cm}
    \Huge ANY\\
    \Huge QUESTIONS?\\
  \end{center}
\end{frame}


\section{References}
\begin{frame}[allowframebreaks]{References}
    \begin{thebibliography}{}

        % Article is the default.
        \setbeamertemplate{bibliography item}[book]
        \bibitem{Randal2018}
         Randal E. Bryant, David R. O'Hallaron
        \newblock \emph{Computer Systems, A prorgammer's perspective}.
        \newblock Pearson, 2018
   

      \setbeamertemplate{bibliography item}[book]
        \bibitem{Hartshorne1977}
          Jonathan Bartlett
        \newblock \emph{Programming from the Ground Up}.
        % \newblock

        \setbeamertemplate{bibliography item}[book]
        \bibitem{Hartshorne1977}
          John R. Levine
        \newblock \emph{Linkers \& Loaders}.
        % \newblock

        \setbeamertemplate{bibliography item}[book]
        \bibitem{Hartshorne1977}
         Oracle
        \newblock \emph{x86 Assembly Language Reference Manual}.
        \newblock 2010
        
        \setbeamertemplate{bibliography item}[online]
        
        \bibitem{HR2018}
          Sandeep.S
        \newblock \emph{GCC-Inline-Assembly-HOWTO}, 2003.
        \newblock \url{https://www.ibiblio.org/gferg/ldp/GCC-Inline-Assembly-HOWTO.html}

        \setbeamertemplate{bibliography item}[online]
      \bibitem{HR201}
        Wikipedia
        \newblock \emph{Application binary interface}, 2020.
        
        \newblock \url{https://en.wikipedia.org/wiki/Application_binary_interface}

        \setbeamertemplate{bibliography item}[online]
      \bibitem{HR201}
        Wikipedia
        \newblock \emph{Calling convention}, 2020.
        \newblock \url{https://en.wikipedia.org/wiki/Calling_convention}
      \end{thebibliography}
    \end{frame}
    
 \section{Thanks}

 \begin{frame}{Thanks}
   \begin{center}
     \vspace{2cm}
     \Huge Thank you ;)!\\
     
     \emph{pqy7172@gmail.com}
   \end{center}
 \end{frame}


\end{document}
