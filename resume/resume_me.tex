\documentclass{wx672article} % $HOME/texmf/tex/latex/wx672article.cls
\usepackage{wx672cjk}



% (c) 2002 Matthew Boedicker <mboedick@mboedick.org> (original author) http://mboedick.org
% (c) 2003-2007 David J. Grant <davidgrant-at-gmail.com> http://www.davidgrant.ca
% (c) 2008 Nathaniel Johnston <nathaniel@nathanieljohnston.com> http://www.nathanieljohnston.com
%
% (c) 2012 Scott Clark <sc932@cornell.edu> cam.cornell.edu/~sc932
%
%This work is licensed under the Creative Commons Attribution-Noncommercial-Share Alike 2.5 License. To view a copy of this license, visit http://creativecommons.org/licenses/by-nc-sa/2.5/ or send a letter to Creative Commons, 543 Howard Street, 5th Floor, San Francisco, California, 94105, USA.

%\documentclass[letterpaper,11pt]{article}
\newlength{\outerbordwidth}
\pagestyle{empty}
\raggedbottom
\raggedright
\usepackage[svgnames]{xcolor}
\usepackage{framed}
\usepackage{tocloft}


%-----------------------------------------------------------
%Edit these values as you see fit

\setlength{\outerbordwidth}{3pt}  % Width of border outside of title bars
\definecolor{shadecolor}{gray}{0.75}  % Outer background color of title bars (0 = black, 1 = white)
\definecolor{shadecolorB}{gray}{0.93}  % Inner background color of title bars


%-----------------------------------------------------------
%Margin setup

\setlength{\evensidemargin}{-0.25in}
\setlength{\headheight}{0in}
\setlength{\headsep}{0in}
\setlength{\oddsidemargin}{-0.25in}
\setlength{\paperheight}{11in}
\setlength{\paperwidth}{8.5in}
\setlength{\tabcolsep}{0in}
\setlength{\textheight}{9.5in}
\setlength{\textwidth}{7in}
\setlength{\topmargin}{-0.3in}
\setlength{\topskip}{0in}
\setlength{\voffset}{0.1in}


%-----------------------------------------------------------
%Custom commands
\newcommand{\resitem}[1]{\item #1 \vspace{-2pt}}
\newcommand{\resheading}[1]{\vspace{8pt}
  \parbox{\textwidth}{\setlength{\FrameSep}{\outerbordwidth}
    \begin{shaded}
\setlength{\fboxsep}{0pt}\framebox[\textwidth][l]{\setlength{\fboxsep}{4pt}\fcolorbox{shadecolorB}{shadecolorB}{\textbf{\sffamily{\mbox{~}\makebox[6.762in][l]{\large #1} \vphantom{p\^{E}}}}}}
    \end{shaded}
  }\vspace{-5pt}
}
\newcommand{\ressubheading}[4]{
\begin{tabular*}{6.5in}{l@{\cftdotfill{\cftsecdotsep}\extracolsep{\fill}}r}
		\textbf{#1} & #2 \\
		\textit{#3} & \textit{#4} \\
\end{tabular*}\vspace{-6pt}}
%-----------------------------------------------------------


\begin{document}

\begin{tabular*}{7in}{l@{\extracolsep{\fill}}r}
  \textbf{\Large 蒲启元} & \textbf{\today} \\
  能模仿着写一个简单的OS。
  & pqy7172@gmail.com \\
  云南昆明 & 电话:18314555392 \\
  年龄:22 & 性别:男
\end{tabular*}
\\


%%%%%%%%%%%%%%%%%%%%%%%%%%%%%%
\resheading{教育背景}
%%%%%%%%%%%%%%%%%%%%%%%%%%%%%%
\vspace{2pt}
\begin{itemize}

\item 学校: 西南林业大学

  \begin{itemize}
  \item 学历:本科。
  \item 时间:2014-9 —— 2018-6。
  \item 专业:计算机科学与技术。
  \item 主修课程:数据结构,操作系统,组成原理,网络等。这几门主要课程均是90分以上。
  \item 其中大三上学期作为学校交换生前往泰国学习。
    
  \end{itemize}
  
\end{itemize}


%%%%%%%%%%%%%%%%%%%%%%%%%%%%%%
\resheading{项目介绍}
%%%%%%%%%%%%%%%%%%%%%%%%%%%%%%

\begin{enumerate}
\item \textbf{名称}:RongOS —  一个简单操作系统的实现。
  \\ \textbf{地址}:https://github.com/Puqiyuan/RongOS。
  \\ \textbf{简介}:本项目从零开始编写了一个简单的操作系统。实现的功能包括多任务、内存管理、
  进程管理、窗口管理等。
  这个项目也是我的毕业设
  计,毕业论文地址https://github.com/Puqiyuan/RongOS/blob/master/doc/thesis/thesis.pdf。

  值得一提的是论文及其中的插图都是我用\LaTeX 排版的,有的图还非常复杂,比如进程管理的插图
  https://github.com/Puqiyuan/RongOS/blob/master/doc/thesis/figs/process-manage.pdf。这些
  精美的插图表达了我对于事情的细致完美追求。
  
  
\item \textbf{名称}:清橙网编程题练习。
  \\ \textbf{地址}:https://github.com/Puqiyuan/Tsinsen\_ACM。
  \\ \textbf{简介}:这个项目是独立自主完成清橙编程题目以训练编程能力,目前所完成的题目都是
  满分通过。其它类似此的题解练习还有:https://github.com/Puqiyuan/URI\_ACM。

  另有两个C程序,高精度(小数点前后各可达999位)浮点计算器
  
  https://github.com/Puqiyuan/High\_Accuracy\_Float\_Calculator/blob/master/Calculator.c
  
  操作系统中银行家算法
  
  https://github.com/Puqiyuan/OS\_Algorithms/blob/master/BankerAlgorithm/Programs/banker.c。
  基本上代表了目前我的最高编程能力。
    

\item \textbf{名称}:UNIX环境高级编程学习记录。
  \\ \textbf{地址}:https://github.com/Puqiyuan/APUE
  \\ \textbf{简介}:学习内容包括文件和目录,标准I/O,系统数据文件和信息,
  进程环境,进程控制,进程关系,信号,线程,线程控制,守护进程,进程通信,网络IPC等。Linux
  下的C开发是我想要从事的方向,未来的职业期望也就是Linux系统/后台开发。
  
\end{enumerate}




%%%%%%%%%%%%%%%%%%%%%%%%%%%%%%
\resheading{自我评价}
%%%%%%%%%%%%%%%%%%%%%%%%%%%%%%
\vspace{2pt}

\hspace*{0.5cm} 
通过大学四年养成了一系列优秀的学习、工作、生活习惯,包括:
\begin{enumerate}
\item 熟练使用Google英文搜索解决各种问题,英文就是我的工作语言。
\item 高效的命令行工作方式。
\item 熟练使用Linux下的成套工具,包括Emacs,Vim,Latex,Org-Mode,Makefile,Github等。
\item 熟练的C编程,具有三年Debian Linux使用配置经验,有一定的Bash Shell编程经验,对Linux下
  的C开发有一定的了解。
\item 行事(编程)条分缕析,条理清楚,逻辑感强,细致追求完美。
\item 坚持每周十或二十公里长跑。耐心与毅力之于技术问题的解决具有决定作用。
\end{enumerate}

\hspace*{0.5cm} 总结起来,大学期间,虽然没有多少实际商业项目经验,但是看的专业书较多,自己独立写的
代码不少,最大的特点是爱钻研好学,在刚开始工作时,虽无多少经验,但拥有这
些基础能力,好的学习习惯,才有足够的后劲来持续学习发展,我相信这些习惯和学习能力会使我行稳
致远。


%%%%%%%%%%%%%%%%%%%%%%%%%%%%%%
\resheading{个人荣誉}
%%%%%%%%%%%%%%%%%%%%%%%%%%%%%%
\vspace{2pt}
\begin{itemize}
\item 2014 — 2015年度校级三好学生。
\item 2015 — 2016年度省级三好学生。
\item 优秀毕业生。
\item 优秀毕业论文(设计)。
\item 2018年云南省大学生计算机作品赛二等奖。
\end{itemize}



\end{document}








%%% Local Variables:
%%% mode: latex
%%% TeX-master: t
%%% End:
